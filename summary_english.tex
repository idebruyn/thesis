
\chapter*{Summary}

All matter we know and see around us is made up of atoms, which consist of negatively charged electrons revolving around a positively charged nucleus. While the electrons are - as far as we currently know - fundamental particles, the nucleus contains protons and neutrons, which are in turn composed of up and down quarks. The theoretical framework that describes all these fundamental particles and their interactions is called the Standard Model of Particle Physics.  While it is an extremely successful theory, multiple unresolved questions and observations cannot be explained by the Standard Model. Cosmological observations, for example, indicate that the known matter described by the Standard Model only contributes 15\% of all the matter in the universe. The remaining matter is observed through gravitational interactions, but is not visible in observations of light at any wavelength, implying it is electrically neutral. Only very little is know about this so-called dark matter, and many theoretical models exist to explain its origin.

Depending on their exact nature, dark matter particles might be produced in high-energy collisions at particle colliders. Many models assume that the dark matter particles interact weakly with ordinary matter, through a new force, making it possible to produce them in the collision of two Standard Model particles. 

This thesis covers two searches for dark matter performed at the Compact Muon Solenoid (CMS) experiment at the CERN Large Hadron Collider. This particle accelerator is currently the largest in the world, and provides proton-proton collisions with a record centre-of-mass energy of 13 TeV at a high collision rate. The CMS detector is a multi-purpose particle detector, used for various precision measurements of the Standard Model and many searches for new physics.

In the first analysis, the dark matter particles are expected to leave the CMS detector undetected as they are neutral and weakly interacting. When they are produced in association with other particles, they can however be observed due to an imbalance of energies measured in the detector, called missing energy. This technique is used in the first dark matter search described in this thesis, called the monojet analysis, where the missing energy is balanced with one or more collimated sprays of particles emerging from the collision, so-called jets. The work in this thesis refined the background prediction and thus increased the sensitivity of the search. No significant excess above the predicted background was observed, setting new, stronger limits on several dark matter models, and excluding a larger part of the available parameter space.

As no observation was made in this first analysis, a more unusual model is studied as well. Instead of looking for weakly interacting massive particles, strongly interacting candidates were considered. These dark matter candidates would be produced in pairs through a new mediating particle, which has a probability to interact with matter that is similar to protons or neutrons. As a result, these particles will leave a signal in the detector that is similar to neutrons, which are electrically neutral as well. The investigated signature is therefore a pair of neutral or so-called trackless jets, which can efficiently be differentiated from the background consisting of charged jets. The result of this search is compatible with the predicted background, and again a part of parameter space was excluded.

To conclude, the two searches covered in this thesis are very complementary, as the missing transverse energy signature used in the monojet search can transform into a trackless jets signature when the interaction probability becomes large enough. Although no sign of new physics was observed, these searches have led to the exclusion of more dark matter scenarios.
% 
% All matter we know and see around us is made up of atoms, which consist of negatively charged electrons revolving around a positively charged nucleus. While the electrons are - as far as we currently know - fundamental particles, the nucleus contains protons and neutrons, which are in turn composed of up and down quarks. Similarly to the electrons, these quarks are elementary particles which do not have a substructure. Some heavier particles can be produced as well, for example at particle colliders, but these eventually decay into the lighter fundamental particles. The theoretical framework that describes all these particles and their interactions is called the Standard Model of Particle Physics.
% 
% This framework has already predicted many experimental results and has survived countless precision measurements so far. While it is an extremely successful theory, multiple unresolved questions and observations cannot be explained by the Standard Model. Gravity, for example, is not incorporated in this model. Moreover, cosmological observations indicate that the known matter described by the Standard Model only contributes 15\% of all the matter in the universe. The remaining  matter is observed through gravitational interactions, but is not visible in observations at any wavelength, implying it is electrically neutral, and it does not interact through any other known force. Only very little is know about this so-called dark matter, and many theoretical models are being constructed to explain its origin. 
% 
% Various searches for dark matter are ongoing, using a multitude of different techniques. Some experiments are looking for dark matter in a direct way by trying to observe it scattering off ordinary matter, while others look for it indirectly by looking for particles or radiation produced in the annihilation of dark matter particles, which is expected to happen in regions with a high dark matter density such as the galactic centre. A third approach consists of producing dark matter particles in high-energy collisions at colliders, and detecting them with particle detectors built around the interaction point.
% 
% This thesis covers two dark matter searches performed at the Compact Muon Solenoid (CMS) experiment at the LHC. This particle accelerator is currently the largest in the world, and it provides proton-proton collisions with a record centre-of-mass energy of \SI{13}{TeV} at high luminosities, i.e. at a high collision rate. The CMS experiment is one of the four experiments located at the four interaction points around the LHC. It is a multi-purpose particle detector, used for various precision measurements of the Standard Model and many searches for new physics.
% 
% Although dark matter does not interact with the ordinary matter through the known interactions described by the Standard Model, many models assume that it interacts weakly through a new force, which is propagated by a new mediator. It is then possible to produce dark matter particles in the collision of two Standard Model particles, through this new mediator. However, the dark matter particles are expected to leave the detector undetected as they are neutral and weakly interacting. When they are produced in association with other particles, they can however be observed due to an imbalance in transverse energy, perpendicular to the proton beams, which is detected as missing energy. This technique is used in the first dark matter search described in this thesis, called the monojet analysis,  where the missing energy is balanced with one or more jets. This already existing analysis was improved by refining the background prediction and reducing the corresponding systematic uncertainties. In this analysis, no significant excess above the predicted background was observed, setting new, stronger limits on several dark matter models, and excluding a larger part of the available parameter space.
% 
% As no observation was made in our first analysis, a more unusual model was studied as well. Instead of looking for weakly interacting massive particles, strongly interacting candidates were considered. These dark matter candidates would be produced in pairs through a new mediator which has a cross section of the order of the interaction of protons and neutrons with matter. Since these particles interact strongly they will leave a signal in the detector, mainly in the calorimeters which have a high material density. However, as they are neutral, they will not leave tracks in the tracking system and can be searched for by looking for trackless jets. The relevant background for this analysis is the production of Standard Model jets, which are usually charged and thus contain tracks. The signal can however efficiently be differentiated from the background by using the charged energy fraction of the jets. The result of this search yields zero events, compatible with the predicted background, and again a part of parameter space was excluded, at larger interaction cross sections.
% 
% To conclude, the two searches covered in this thesis are very complementary, as the missing transverse energy signature used in the monojet search can transform into a trackless jets signature when the interaction cross section becomes large enough. Although no excess was observed, these searches help exclude more dark matter scenarios, and in addition the trackless jets search allows to learn more about the detector, which was not built to look for this type of signature. This type of searches provides more insight on how to investigate new models that give rise to unusual signatures.
