
\chapter*{Summary}

All the matter we know and see around us is made up of atoms which consist of electrons revolving around a nucleus containing protons and neutrons.  These nucleons are in turn composed of up and down quarks, which – as far as we currently know – are fundamental particles like the electrons. Some heavier particles can be produced as well, amongst others at particle colliders, but these eventually decay into the lighter fundamental particles. The theoretical framework that describes all these particles and their interactions is called the Standard Model of Particle Physics.

This consistent framework has already predicted many experimental results and has survived countless precision measurements so far. While it seems to cover pretty much everything, multiple unresolved questions and observations cannot be explained by the Standard Model. Gravity, for example, is not incorporated in this model. Moreover, cosmological observations indicate that the known matter described by the Standard Model only contributes 15\% of all the matter in the universe. The remainder of the matter is observed through gravitational interactions, but is not visible via any other known force, such as electromagnetism. Only very little is know about this so-called dark matter, and many theoretical models are begin constructed to explain it’s origin. 

Furthermore, various searches for dark matter are ongoing, using a multitude of different techniques. Some experiments are looking for dark matter in a direct way by trying to observe it scattering off ordinary matter, while others look for it indirectly by looking for particles or radiation produced in the annihilation of dark matter particles. A third type of experiments is based at particle colliders, such as the Large Hadron Collider (LHC) at CERN, and tries to detect dark matter particles by producing them in high-energy collisions.

This thesis covers two dark matter searches performed at the Compact Muon Solenoid (CMS) experiment at the LHC. This particle accelerator is currently the largest in the world, and provides proton-proton collisions with a centre-of-mass energy of 13 TeV with a high luminosity. The CMS experiment is one of the four experiments located at the four interaction points around the LHC. It is a multi-purpose detector where various precision measurements of the Standard Model and searches for new physics are being performed. In this thesis, the data taken in 2015 and 2016 was used.

Although the dark matter does not interact with the ordinary matter through the known interactions described by the Standard Model, it is in general assumed to interact weakly through a new mediator. This would make it possible to produce dark matter particles in the collision of two Standard Model particles. However, they are expected to leave the detector undetected as they are neutral and weakly interacting. When they are produced in association with other particles, they can however be observed due to an imbalance in transverse energy generating missing transverse energy. This technique is used in the first dark matter search described in this thesis, called the monojet analysis,  where missing transverse energy is being looked for along with one or more jets. This already existing analysis was greatly improved by improving the background prediction and reducing the corresponding systematic uncertainties on it. No significant excess above the predicted background was observed, however, and new, stronger limits are set on several dark matter models, excluding a larger part of the phase space.

As no observation was made so far, a more unusual model was studied as well. Instead of weakly interacting massive particles as dark matter candidates, strongly interacting candidates were considered. These dark matter candidates are produced in pairs through a new mediator which has a cross section of the order of the hadronic interaction. Since these neutral particles interact strongly they will leave a signal in the detector, mainly in the calorimeters which have a high material density. As a result they can be searched for by looking for trackless jets. The relevant background for this analysis is the production of Standard Model jets, which are usually charged and thus contain tracks. The signal can however efficiently be extracted using the charged energy fraction of the jets. Zero events were found in the defined signal region, and again a part of phase space was excluded, at larger interaction cross sections.

To conclude, the two searches covered in this thesis are very complementary, as the missing transverse energy signature used in the monojet search can transform into a trackless jets signature when the interaction cross section becomes large enough. Although no excess was observed, these searches help exclude more dark matter scenarios, and in addition the trackless jets search allows to learn more about the detector, which was not built to look for this type of signals. This allows to gain more experience on how to search for new models giving rise to unusual signatures.