\graphicspath{{chapt_dutch/}{intro/}{theory/}{detector/}{reconstruction/}}

% Header
\renewcommand\evenpagerightmark{{\scshape\small Chapter 1}}
\renewcommand\oddpageleftmark{{\scshape\small Introduction}}

\hyphenation{}

\chapter{Introduction}
\label{ch:intro}

%% Introduction
%%%%%%%%%%%%%%%

\begin{flushleft} 
\textit{There is a theory which states that if ever anyone discovers exactly what the Universe is for and why it is here, it will instantly disappear and be replaced by something even more bizarre and inexplicable. There is another theory which states that this has already happened.}
\end{flushleft}
\begin{flushright}
-- Douglas Adams, \textit{The Hitchhiker's Guide to the Galaxy}
\end{flushright}

Nevertheless, humankind is still trying to understand the most fundamental aspects of our universe, by studying the fundamental particles matter is made of and the interactions between them. The story so far has been summarised in a theory called the Standard Model of Particle Physics. This theory has been extensively tested and has already predicted many experimental observations. However, it cannot explain the full story, and some pieces remain missing. Gravity, for example, is not incorporated in the Standard Model. Similarly, it cannot explain the observed neutrino masses or the matter-antimatter asymmetry.

Another mystery stems from a series of cosmological observations made during the last century. These observations are based on gravitational effects, such as measurements of the rotation curves of galaxies~\cite{Begeman:1991iy} and gravitational lensing~\cite{Belokurov:2008pu}, and on the analysis of the \acf{CMB}~\cite{Smoot:1992td,Komatsu:2010fb,Ade:2013zuv}. The collected evidence shows that there is matter in the universe, which is not visible from measurements at any wavelength of the electromagnetic spectrum. This so-called dark matter was found to constitute about 85\% of the matter in the universe, which means that the ordinary matter described by the Standard Model only accounts for 15\% of all matter. So far, only very little is known about the dark matter, as it does not interact through any of the forces included in the Standard Model, and has only been observed through gravitational interaction at large scales. Many theoretical models therefore exist, that try to model this unknown type of matter. These theories generally assume that this form of matter is composed of particles, just as the known matter, and describe a new type of interaction through which these dark matter particles interact with the Standard Model particles.

If the dark matter indeed interacts with ordinary matter through a new force, mediated by a new particle, it can be searched for, and many existing theories can be tested. A myriad of experiments are currently looking for dark matter, and can be divided into three categories. Firstly, direct detection experiments take advantage of the dark matter particles that should be present in a halo permeating our galaxy and try to measure the recoil of nuclei generated by dark matter particles passing through the Earth and scattering off the ordinary matter. The detectors used for this type of experiment are mostly located underground and are well shielded from radiation, though a few are airborne or space experiments. Indirect detection experiments on the other hand look for particles or radiation coming from the annihilation of dark matter particles in dense regions such as the galactic centre. These searches are studying gamma rays, neutrinos, electrons and positrons, or radio emissions. Finally, dark matter particles could potentially also be produced and detected at particle colliders. One of the direct detection experiments observed evidence pointing to the existence dark matter particles, but so far no conclusive observations have been made.

At collider experiments, such as \acs{ATLAS} and \acs{CMS}, dark matter candidates are often looked for by focusing on missing energy. Indeed, if the dark dark matter is assumed to interact weakly with the ordinary matter, it will be able to leave the detector unnoticed. However, these so-called \acfp{WIMP}, can be detected when they recoil against another object. Some examples of such collider searches are the monophoton, monojet, monolepton, and mono-Higgs analyses, categorised based on the signature in the detector. Additionally, different signatures are obtained when the dark matter is for example produced in a cascade of decays. Also resonances in e.g. the dijet mass spectrum are looked for, as this could indicate the existence of a new dark matter mediator. In general, more and more analyses are adding dark matter interpretations to their results. This thesis describes two searches for dark matter performed using data from high-energy proton-proton collisions produced at a centre-of-mass energy of \SI{13}{TeV} and recorded with the \acs{CMS} detector. The first analysis is the so-called the monojet analysis, which investigates the existence of \acsp{WIMP} as dark matter candidates. Conversely, the second search looks for dark matter in the form of \acfp{SIMP}.

In Chapter~\ref{ch:theory}, an overview of the Standard Model is given, as well as a short description of a few of its shortcomings. Furthermore, a summary of the existing evidence for dark matter, together with a concise review of popular dark models and a brief description of the operational or developing dark matter experiments are given. As this thesis covers dark matter searches that are performed using the \acs{CMS} detector, located at one of the collision points of the \acs{LHC} at \acs{CERN}, more details concerning this accelerator and particle detector are summarised in Chapter~\ref{ch3}. In Chapter~\ref{ch:reconstruction}, the procedure to reconstruct the collisions occurring inside the detector is detailed, as well as the necessary simulations of the predicted signal, which are needed in order to design a search for a particular dark matter candidate and to tune the analysis to the expected signature in the detector. The required techniques for this are described, with more details on the specific simulations needed for the searches covered in this thesis. The two complementary dark matter searches are described in Chapters~\ref{ch:monojet} and~\ref{ch:SIMPs}.

The monojet analysis, covered in Chapter~\ref{ch:monojet}, is one of the flagship analyses which were designed to quickly detect potential dark matter candidates, for a broad range of models. I contributed to this analysis by improving the prediction of the main background, coming from the invisible decay of $Z$ bosons into neutrinos, produced in association with one or more jets. The used strategy for the background estimation is detailed in Section~\ref{sec:bkgd} and the resulting impact on the sensitivity of the analysis is shown in Section~\ref{sec:improvement}. In the \ac{SIMP} analysis, described in Chapter~\ref{ch:SIMPs}, the dark matter candidates and the Standard Model particles interact strongly through a new force, carried by a new, light mediator. The signature therefore does not consist of missing energy, but instead trackless jets are created due to the interaction of these neutral \acsp{SIMP} in the dense material of the calorimeters in the detector. First, a phenomenological study of the dark matter model was performed and published~\cite{Daci:2015hca}, and subsequently the search was carried out using data collected by \acs{CMS} in 2016. This work, together with my contribution to the monojet analysis are the main topics of my PhD research. The monojet search provides new, stronger limits on \acs{WIMP} dark matter candidates, while the trackless jets search rules out a new dark matter model which had not been tested at colliders yet.

{\color{red} mention something about tracker work? (Kevin mentions b-tagging)}

% \renewcommand*{\thesection}{\thechapter.\arabic{section}}       % reset again to chaptnum.sectnum

\clearpage{\pagestyle{empty}\cleardoublepage}
