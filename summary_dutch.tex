
\chapter*{Samenvatting}

Alle materie die we rondom ons zien, is opgemaakt uit atomen, bestaande uit negatief geladen elektronen die rond een positief geladen kern cirkelen. Hoewel de elektronen – voor zover onze huidige kennis strekt – elementaire deeltjes zijn, bevat de kern van een atoom protonen en neutronen, die op hun beurt bestaan uit up en down quarks. Al deze deeltjes en hun onderlinge wisselwerking worden beschreven in een theorie die gekend staat als het Standaard Model van de Deeltjesfysica. Desondanks kan deze uiterst succesvolle theorie niet alle waargenomen fenomenen verklaren. Verschillende kosmologische waarnemingen tonen namelijk aan dat de gekende materie die door het Standaard Model beschreven wordt maar 15\% van de totale materie in het universum beslaat. De overige materie kan waargenomen worden door middel van  zwaartekrachteffecten, maar is niet zichtbaar via waarnemingen gebaseerd op licht van eender welke golflengte. Dit wijst erop dat deze zogenaamde donkere materie ongeladen is. Verder is er over deze materie bitter weinig geweten, maar bestaan er talloze theoretische modellen die de oorsprong ervan proberen te verklaren.

Afhankelijk van de aard van de donkere materie deeltjes, kunnen deze geproduceerd worden in hoogenergetische botsingen die plaatsvinden in deeltjesversnellers. Vele modellen nemen aan dat de donkere materie deeltjes zwak met de gewone materie interageren, via een nieuwe, onbekende kracht. Dit maakt het mogelijk om deze deeltjes te produceren door deeltjes van de gekende materie aan een hoge snelheid tegen elkaar te laten botsen.

Deze thesis behandelt twee zoektochten naar donkere materie, die uitgevoerd werden aan het Compact Muon Solenoid (CMS) experiment dat zich aan de Large Hadron Collider (LHC) in het CERN bevindt. De LHC is momenteel ‘s werelds grootste en krachtigste deeltjesversneller en laat protonen aan een hoge frequentie tegen elkaar botsen, met een massamiddelpuntsenergie van 13 TeV. De CMS detector is veelzijdig en wordt zowel voor precieze testen van het Standaard Model als voor onderzoek naar nieuwe fysica gebruikt.

In de eerste analyse worden de donkere materie deeltjes verondersteld de CMS detector ongezien te verlaten, aangezien ze neutraal zijn en zwak interageren. Wanneer ze samen met andere deeltjes geproduceerd worden, kunnen deze evenwel waargenomen worden door een onevenwicht in de gemeten energieën, wat leidt tot ontbrekende energie in de detector. Deze methode wordt in de zogenaamde monojet analyse gebruikt door te zoeken naar een combinatie van ontbrekende energie en gecollimeerde bundels van deeltjes, zogenaamde jets. Het werk in deze thesis heeft tot een nauwkeurigere voorspelling van de achtergrond geleid en heeft het bekomen resultaat aanzienlijk verbeterd. Er werd geen nieuwe fysica waargenomen boven op de voorspelde achtergrond, en nieuwe, strengere beperkingen werden op deze manier aan de beschouwde modellen opgelegd.

In het tweede deel van deze thesis wordt een enigszins ongewoon model bestudeerd. In dit geval wordt er gezocht naar sterk interagerende donkere materie deeltjes, in tegenstelling tot zwak interagerende deeltjes. Deze donkere materie deeltjes zouden in paren geproduceerd worden via een nieuw krachtdragend deeltje, dat in gelijke mate met de gekende materie interageert als protonen en neutronen. Het signaal dat deze deeltjes in de detector achterlaten lijkt bijgevolg sterk op dat van neutronen, aangezien deze ook ongeladen zijn. Er wordt dus gezocht naar een paar neutrale jets, zogenaamde \textit{trackless jets}, die gemakkelijk onderscheiden kunnen worden van de achtergrond bestaande uit geladen jets. Het resultaat van dit onderzoek is volledig compatibel met de voorspelde achtergrond, waardoor het beschouwde model uitgesloten wordt.

De twee onderzochte scenario's vullen elkaar aan, daar de ontbrekende energie die in de monojet analyse gebruikt wordt in neutrale jets omgevormd kan worden wanneer de interactiewaarschijnlijkheid van de deeltjes groot genoeg wordt. Hoewel er geen teken van nieuwe fysica waargenomen werd, leiden deze zoektochten tot het uitsluiten van een aantal donkere materie modellen.