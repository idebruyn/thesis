\chapter*{Author's Contribution}

Two different dark matter searches are presented in this thesis. The monojet analysis was performed in collaboration with groups from UCSD, Rome, Cornell, MIT, and CERN. After a more general description of the analysis, the contributions of the author are explained in more detail. The specific contributions concern the improvement of the theoretical and statistical uncertainties on the main background estimation, including the generation and validation of new simulated samples. This work resulted in the following publication:

\begin{center}
\textit{CMS Collaboration, Search for dark matter produced with an energetic jet or a hadronically decaying W or Z boson at $\sqrt{s}$ = 13 TeV, JHEP, vol. 7, p. 14, 2017}
\end{center}

For the second search, the author first performed a study together with colleagues from VUB and ULB. In this work the existing constraints on the considered model are summarised, the feasibility study of a search is performed, and the translation into the direct detection plane is implemented in order to compare to direct detection and other experiments. The author has been the main contributor to the generation and simulation of the expected signal, the design of the analysis, and the study of the reachable sensitivity at the \acs{LHC}. The corresponding publication is:

\begin{center}
\textit{N. Daci, I. De Bruyn, S. Lowette, M.H.G. Tytgat, and B. Zaldivar, Simplified SIMPs and the LHC, JHEP, vol. 11, p. 108, 2015}
\end{center}

Based on the positive results of this study, the analysis was then performed using CMS data. The author is the main contributor to this analysis, which will be published in the near future. The author's work includes the generation of the signal, the design and improvement of the analysis strategy, the validation of the used method, and the acquisition of the final results.

\vspace{.5cm}

Besides the work on the dark matter searches described in this thesis, the author also contributed in several ways to \acs{CMS} experiment, more specifically related to the tracker and the data taking itself. During the shutdown preceding \acs{LHC} Run~2, the author participated in the preparations to operate the tracker at a colder coolant temperature and to the subsequent re-commissioning. Afterwards, during data taking, the author continued to be involved by taking on shifts in the control room and additionally contributed to the development and maintenance of the tracker \acf{DQM} framework, which is used to spot problems in data taking and to certify the recorded data. Finally, the author took up more responsibilities by becoming co-convener of the tracker \acs{DQM} group.